% Options for packages loaded elsewhere
\PassOptionsToPackage{unicode}{hyperref}
\PassOptionsToPackage{hyphens}{url}
%
\documentclass[
]{article}
\usepackage{amsmath,amssymb}
\usepackage{iftex}
\ifPDFTeX
  \usepackage[T1]{fontenc}
  \usepackage[utf8]{inputenc}
  \usepackage{textcomp} % provide euro and other symbols
\else % if luatex or xetex
  \usepackage{unicode-math} % this also loads fontspec
  \defaultfontfeatures{Scale=MatchLowercase}
  \defaultfontfeatures[\rmfamily]{Ligatures=TeX,Scale=1}
\fi
\usepackage{lmodern}
\ifPDFTeX\else
  % xetex/luatex font selection
\fi
% Use upquote if available, for straight quotes in verbatim environments
\IfFileExists{upquote.sty}{\usepackage{upquote}}{}
\IfFileExists{microtype.sty}{% use microtype if available
  \usepackage[]{microtype}
  \UseMicrotypeSet[protrusion]{basicmath} % disable protrusion for tt fonts
}{}
\makeatletter
\@ifundefined{KOMAClassName}{% if non-KOMA class
  \IfFileExists{parskip.sty}{%
    \usepackage{parskip}
  }{% else
    \setlength{\parindent}{0pt}
    \setlength{\parskip}{6pt plus 2pt minus 1pt}}
}{% if KOMA class
  \KOMAoptions{parskip=half}}
\makeatother
\usepackage{xcolor}
\usepackage[margin=1in]{geometry}
\usepackage{color}
\usepackage{fancyvrb}
\newcommand{\VerbBar}{|}
\newcommand{\VERB}{\Verb[commandchars=\\\{\}]}
\DefineVerbatimEnvironment{Highlighting}{Verbatim}{commandchars=\\\{\}}
% Add ',fontsize=\small' for more characters per line
\usepackage{framed}
\definecolor{shadecolor}{RGB}{248,248,248}
\newenvironment{Shaded}{\begin{snugshade}}{\end{snugshade}}
\newcommand{\AlertTok}[1]{\textcolor[rgb]{0.94,0.16,0.16}{#1}}
\newcommand{\AnnotationTok}[1]{\textcolor[rgb]{0.56,0.35,0.01}{\textbf{\textit{#1}}}}
\newcommand{\AttributeTok}[1]{\textcolor[rgb]{0.13,0.29,0.53}{#1}}
\newcommand{\BaseNTok}[1]{\textcolor[rgb]{0.00,0.00,0.81}{#1}}
\newcommand{\BuiltInTok}[1]{#1}
\newcommand{\CharTok}[1]{\textcolor[rgb]{0.31,0.60,0.02}{#1}}
\newcommand{\CommentTok}[1]{\textcolor[rgb]{0.56,0.35,0.01}{\textit{#1}}}
\newcommand{\CommentVarTok}[1]{\textcolor[rgb]{0.56,0.35,0.01}{\textbf{\textit{#1}}}}
\newcommand{\ConstantTok}[1]{\textcolor[rgb]{0.56,0.35,0.01}{#1}}
\newcommand{\ControlFlowTok}[1]{\textcolor[rgb]{0.13,0.29,0.53}{\textbf{#1}}}
\newcommand{\DataTypeTok}[1]{\textcolor[rgb]{0.13,0.29,0.53}{#1}}
\newcommand{\DecValTok}[1]{\textcolor[rgb]{0.00,0.00,0.81}{#1}}
\newcommand{\DocumentationTok}[1]{\textcolor[rgb]{0.56,0.35,0.01}{\textbf{\textit{#1}}}}
\newcommand{\ErrorTok}[1]{\textcolor[rgb]{0.64,0.00,0.00}{\textbf{#1}}}
\newcommand{\ExtensionTok}[1]{#1}
\newcommand{\FloatTok}[1]{\textcolor[rgb]{0.00,0.00,0.81}{#1}}
\newcommand{\FunctionTok}[1]{\textcolor[rgb]{0.13,0.29,0.53}{\textbf{#1}}}
\newcommand{\ImportTok}[1]{#1}
\newcommand{\InformationTok}[1]{\textcolor[rgb]{0.56,0.35,0.01}{\textbf{\textit{#1}}}}
\newcommand{\KeywordTok}[1]{\textcolor[rgb]{0.13,0.29,0.53}{\textbf{#1}}}
\newcommand{\NormalTok}[1]{#1}
\newcommand{\OperatorTok}[1]{\textcolor[rgb]{0.81,0.36,0.00}{\textbf{#1}}}
\newcommand{\OtherTok}[1]{\textcolor[rgb]{0.56,0.35,0.01}{#1}}
\newcommand{\PreprocessorTok}[1]{\textcolor[rgb]{0.56,0.35,0.01}{\textit{#1}}}
\newcommand{\RegionMarkerTok}[1]{#1}
\newcommand{\SpecialCharTok}[1]{\textcolor[rgb]{0.81,0.36,0.00}{\textbf{#1}}}
\newcommand{\SpecialStringTok}[1]{\textcolor[rgb]{0.31,0.60,0.02}{#1}}
\newcommand{\StringTok}[1]{\textcolor[rgb]{0.31,0.60,0.02}{#1}}
\newcommand{\VariableTok}[1]{\textcolor[rgb]{0.00,0.00,0.00}{#1}}
\newcommand{\VerbatimStringTok}[1]{\textcolor[rgb]{0.31,0.60,0.02}{#1}}
\newcommand{\WarningTok}[1]{\textcolor[rgb]{0.56,0.35,0.01}{\textbf{\textit{#1}}}}
\usepackage{graphicx}
\makeatletter
\def\maxwidth{\ifdim\Gin@nat@width>\linewidth\linewidth\else\Gin@nat@width\fi}
\def\maxheight{\ifdim\Gin@nat@height>\textheight\textheight\else\Gin@nat@height\fi}
\makeatother
% Scale images if necessary, so that they will not overflow the page
% margins by default, and it is still possible to overwrite the defaults
% using explicit options in \includegraphics[width, height, ...]{}
\setkeys{Gin}{width=\maxwidth,height=\maxheight,keepaspectratio}
% Set default figure placement to htbp
\makeatletter
\def\fps@figure{htbp}
\makeatother
\setlength{\emergencystretch}{3em} % prevent overfull lines
\providecommand{\tightlist}{%
  \setlength{\itemsep}{0pt}\setlength{\parskip}{0pt}}
\setcounter{secnumdepth}{-\maxdimen} % remove section numbering
\ifLuaTeX
  \usepackage{selnolig}  % disable illegal ligatures
\fi
\usepackage{bookmark}
\IfFileExists{xurl.sty}{\usepackage{xurl}}{} % add URL line breaks if available
\urlstyle{same}
\hypersetup{
  pdftitle={HW1},
  pdfauthor={Kwabena Bayity},
  hidelinks,
  pdfcreator={LaTeX via pandoc}}

\title{HW1}
\author{Kwabena Bayity}
\date{}

\begin{document}
\maketitle

\begin{Shaded}
\begin{Highlighting}[]
\CommentTok{\# Load necessary libraries}
\FunctionTok{library}\NormalTok{(readr)}
\FunctionTok{library}\NormalTok{(dplyr)}
\FunctionTok{library}\NormalTok{(latexpdf)}

\CommentTok{\# Load the data}
\NormalTok{senic }\OtherTok{\textless{}{-}} \FunctionTok{read.csv}\NormalTok{(}\StringTok{"\textasciitilde{}/Desktop/MY Purdue/Summer 2025/STATS/Scripts/senic.csv"}\NormalTok{)}

\CommentTok{\# Preview the data}
\FunctionTok{head}\NormalTok{(senic)}
\end{Highlighting}
\end{Shaded}

\begin{verbatim}
##   id length  age infection cultur  xray bed affiliat region region_recode
## 1  1   7.13 55.7       4.1    9.0  39.6 279        2      4             4
## 2  2   8.82 58.2       1.6    3.8  51.7  80        2      2             1
## 3  3   8.34 56.9       2.7    8.1  74.0 107        2      3             3
## 4  4   8.95 53.7       5.6   18.9 122.8 147        2      4             4
## 5  5  11.20 56.5       5.7   34.5  88.9 180        2      1             2
## 6  6   9.76 50.9       5.1   21.9  97.0 150        2      2             1
##   patient nurse facility
## 1     207   241       60
## 2      51    52       40
## 3      82    54       20
## 4      53   148       40
## 5     134   151       40
## 6     147   106       40
\end{verbatim}

\begin{Shaded}
\begin{Highlighting}[]
\CommentTok{\# View the structure of your dataset}
\FunctionTok{str}\NormalTok{(senic)}
\end{Highlighting}
\end{Shaded}

\begin{verbatim}
## 'data.frame':    113 obs. of  13 variables:
##  $ id           : int  1 2 3 4 5 6 7 8 9 10 ...
##  $ length       : num  7.13 8.82 8.34 8.95 11.2 ...
##  $ age          : num  55.7 58.2 56.9 53.7 56.5 50.9 57.8 45.7 48.2 56.3 ...
##  $ infection    : num  4.1 1.6 2.7 5.6 5.7 5.1 4.6 5.4 4.3 6.3 ...
##  $ cultur       : num  9 3.8 8.1 18.9 34.5 21.9 16.7 60.5 24.4 29.6 ...
##  $ xray         : num  39.6 51.7 74 122.8 88.9 ...
##  $ bed          : int  279 80 107 147 180 150 186 640 182 85 ...
##  $ affiliat     : int  2 2 2 2 2 2 2 1 2 2 ...
##  $ region       : int  4 2 3 4 1 2 3 2 3 1 ...
##  $ region_recode: int  4 1 3 4 2 1 3 1 3 2 ...
##  $ patient      : int  207 51 82 53 134 147 151 399 130 59 ...
##  $ nurse        : int  241 52 54 148 151 106 129 360 118 66 ...
##  $ facility     : num  60 40 20 40 40 40 40 60 40 40 ...
\end{verbatim}

\begin{Shaded}
\begin{Highlighting}[]
\CommentTok{\#Define Variables}

\CommentTok{\# Assign variables}
\NormalTok{X }\OtherTok{\textless{}{-}}\NormalTok{ senic}\SpecialCharTok{$}\NormalTok{infection    }\CommentTok{\# Independent Variable: Risk of Infection}
\NormalTok{Y }\OtherTok{\textless{}{-}}\NormalTok{ senic}\SpecialCharTok{$}\NormalTok{length  }\CommentTok{\# Dependent Variable: the time it takes or duration}

\CommentTok{\# Check for missing values}
\FunctionTok{sum}\NormalTok{(}\FunctionTok{is.na}\NormalTok{(X)); }\FunctionTok{sum}\NormalTok{(}\FunctionTok{is.na}\NormalTok{(Y))}
\end{Highlighting}
\end{Shaded}

\begin{verbatim}
## [1] 0
\end{verbatim}

\begin{verbatim}
## [1] 0
\end{verbatim}

\begin{Shaded}
\begin{Highlighting}[]
\CommentTok{\# Part 1: Compute Regression Coefficients Step by Step}

\CommentTok{\# Step 1: Compute means}
\NormalTok{X\_bar }\OtherTok{\textless{}{-}} \FunctionTok{mean}\NormalTok{(X)}
\NormalTok{Y\_bar }\OtherTok{\textless{}{-}} \FunctionTok{mean}\NormalTok{(Y)}


\CommentTok{\# Step 2: Compute b1 (slope)}
\NormalTok{numerator }\OtherTok{\textless{}{-}} \FunctionTok{sum}\NormalTok{((X }\SpecialCharTok{{-}}\NormalTok{ X\_bar) }\SpecialCharTok{*}\NormalTok{ (Y }\SpecialCharTok{{-}}\NormalTok{ Y\_bar))}
\NormalTok{denominator }\OtherTok{\textless{}{-}} \FunctionTok{sum}\NormalTok{((X }\SpecialCharTok{{-}}\NormalTok{ X\_bar)}\SpecialCharTok{\^{}}\DecValTok{2}\NormalTok{)}
\NormalTok{b1 }\OtherTok{\textless{}{-}}\NormalTok{ numerator }\SpecialCharTok{/}\NormalTok{ denominator}

\CommentTok{\# Step 3: Compute b0 (intercept)}
\NormalTok{b0 }\OtherTok{\textless{}{-}}\NormalTok{ Y\_bar }\SpecialCharTok{{-}}\NormalTok{ b1 }\SpecialCharTok{*}\NormalTok{ X\_bar}


\CommentTok{\# Step 4: Compute fitted values and residuals}
\NormalTok{Y\_hat }\OtherTok{\textless{}{-}}\NormalTok{ b0 }\SpecialCharTok{+}\NormalTok{ b1 }\SpecialCharTok{*}\NormalTok{ X}
\NormalTok{residuals }\OtherTok{\textless{}{-}}\NormalTok{ Y }\SpecialCharTok{{-}}\NormalTok{ Y\_hat}


\CommentTok{\# Step 5: Compute SSE, MSE, SST, SSR}
\NormalTok{SSE }\OtherTok{\textless{}{-}} \FunctionTok{sum}\NormalTok{(residuals}\SpecialCharTok{\^{}}\DecValTok{2}\NormalTok{)}
\NormalTok{MSE }\OtherTok{\textless{}{-}}\NormalTok{ SSE }\SpecialCharTok{/}\NormalTok{ (}\FunctionTok{length}\NormalTok{(Y) }\SpecialCharTok{{-}} \DecValTok{2}\NormalTok{)}
\NormalTok{SST }\OtherTok{\textless{}{-}} \FunctionTok{sum}\NormalTok{((Y }\SpecialCharTok{{-}}\NormalTok{ Y\_bar)}\SpecialCharTok{\^{}}\DecValTok{2}\NormalTok{)}
\NormalTok{SSR }\OtherTok{\textless{}{-}} \FunctionTok{sum}\NormalTok{((Y\_hat }\SpecialCharTok{{-}}\NormalTok{ Y\_bar)}\SpecialCharTok{\^{}}\DecValTok{2}\NormalTok{)}

\CommentTok{\# Step 6: Verify decomposition}
\NormalTok{SST\_check }\OtherTok{\textless{}{-}}\NormalTok{ SSR }\SpecialCharTok{+}\NormalTok{ SSE}
\end{Highlighting}
\end{Shaded}

\begin{Shaded}
\begin{Highlighting}[]
\CommentTok{\# Predicted values and residuals}
\NormalTok{Y\_hat }\OtherTok{\textless{}{-}}\NormalTok{ b0 }\SpecialCharTok{+}\NormalTok{ b1 }\SpecialCharTok{*}\NormalTok{ X}
\NormalTok{residuals }\OtherTok{\textless{}{-}}\NormalTok{ Y }\SpecialCharTok{{-}}\NormalTok{ Y\_hat}

\CommentTok{\# Sum of Squares}
\NormalTok{SSE }\OtherTok{\textless{}{-}} \FunctionTok{sum}\NormalTok{(residuals}\SpecialCharTok{\^{}}\DecValTok{2}\NormalTok{)}
\NormalTok{SST }\OtherTok{\textless{}{-}} \FunctionTok{sum}\NormalTok{((Y }\SpecialCharTok{{-}}\NormalTok{ Y\_bar)}\SpecialCharTok{\^{}}\DecValTok{2}\NormalTok{)}
\NormalTok{SSR }\OtherTok{\textless{}{-}} \FunctionTok{sum}\NormalTok{((Y\_hat }\SpecialCharTok{{-}}\NormalTok{ Y\_bar)}\SpecialCharTok{\^{}}\DecValTok{2}\NormalTok{)}

\CommentTok{\# Mean Squared Error}
\NormalTok{n }\OtherTok{\textless{}{-}} \FunctionTok{length}\NormalTok{(Y)}
\NormalTok{df }\OtherTok{\textless{}{-}}\NormalTok{ n }\SpecialCharTok{{-}} \DecValTok{2}
\NormalTok{MSE }\OtherTok{\textless{}{-}}\NormalTok{ SSE }\SpecialCharTok{/}\NormalTok{ df}

\CommentTok{\# show results}
\NormalTok{SSE; SST; SSR}
\end{Highlighting}
\end{Shaded}

\begin{verbatim}
## [1] 292.7645
\end{verbatim}

\begin{verbatim}
## [1] 409.2104
\end{verbatim}

\begin{verbatim}
## [1] 116.4459
\end{verbatim}

\begin{Shaded}
\begin{Highlighting}[]
\NormalTok{MSE}
\end{Highlighting}
\end{Shaded}

\begin{verbatim}
## [1] 2.637518
\end{verbatim}

\begin{Shaded}
\begin{Highlighting}[]
\CommentTok{\# double check if SST = SSR + SSE}
\FunctionTok{all.equal}\NormalTok{(SST, SSR }\SpecialCharTok{+}\NormalTok{ SSE)}
\end{Highlighting}
\end{Shaded}

\begin{verbatim}
## [1] TRUE
\end{verbatim}

\begin{Shaded}
\begin{Highlighting}[]
\CommentTok{\# Output results}
\FunctionTok{list}\NormalTok{(}
  \AttributeTok{X\_bar =}\NormalTok{ X\_bar,}
  \AttributeTok{Y\_bar =}\NormalTok{ Y\_bar,}
  \AttributeTok{sum\_xy\_dev =}\NormalTok{ numerator,}
  \AttributeTok{sum\_x\_dev\_sq =}\NormalTok{ denominator,}
  \AttributeTok{b1 =}\NormalTok{ b1,}
  \AttributeTok{b0 =}\NormalTok{ b0,}
  \AttributeTok{SSE =}\NormalTok{ SSE,}
  \AttributeTok{MSE =}\NormalTok{ MSE,}
  \AttributeTok{SST =}\NormalTok{ SST,}
  \AttributeTok{SSR =}\NormalTok{ SSR,}
  \AttributeTok{SST\_equals\_SSR\_plus\_SSE =} \FunctionTok{all.equal}\NormalTok{(SST, SSR }\SpecialCharTok{+}\NormalTok{ SSE)}
\NormalTok{)}
\end{Highlighting}
\end{Shaded}

\begin{verbatim}
## $X_bar
## [1] 4.354867
## 
## $Y_bar
## [1] 9.648319
## 
## $sum_xy_dev
## [1] 153.1334
## 
## $sum_x_dev_sq
## [1] 201.3798
## 
## $b1
## [1] 0.7604209
## 
## $b0
## [1] 6.336787
## 
## $SSE
## [1] 292.7645
## 
## $MSE
## [1] 2.637518
## 
## $SST
## [1] 409.2104
## 
## $SSR
## [1] 116.4459
## 
## $SST_equals_SSR_plus_SSE
## [1] TRUE
\end{verbatim}

\section{I already have:}\label{i-already-have}

\begin{verbatim}
b1b1​: the slope

SSE: sum of squared errors

MSE: mean squared error

∑(Xi−Xˉ)2∑(Xi​−Xˉ)2: the denominator used in slope calculation
\end{verbatim}

\section{Question 2:}\label{question-2}

\begin{enumerate}
\def\labelenumi{(\arabic{enumi})}
\setcounter{enumi}{7}
\tightlist
\item
  In order to estimate the linear impact of X on Y, at a confidence of
  (1-α)\%, you should use the critical value, or the t value denoted as
  t(\textbf{\emph{, }}\_), which has a value of \_\_\_\_ (use basic R
  function or Excel for the exact value), at α=0.1, and \_\_\_\_\_at
  α=0.05. The standard error of the estimation s\{b\_1 \}=
  \_\_\_\_\_\_\_\_\_\_\_\_\_\_\_(formula)=\_\_\_\_\_\_\_\_(value). The
  margin error, or t*SE, of the confidence interval is
  \_\_\_\_\_\_\_\_\_ at α=0.1, and \_\_\_\_\_at α=0.05.
\end{enumerate}

\section{Solution}\label{solution}

\begin{Shaded}
\begin{Highlighting}[]
\CommentTok{\# Degrees of freedom}
\NormalTok{n }\OtherTok{\textless{}{-}} \FunctionTok{length}\NormalTok{(Y)}
\NormalTok{df }\OtherTok{\textless{}{-}}\NormalTok{ n }\SpecialCharTok{{-}} \DecValTok{2}

\CommentTok{\# Critical t{-}values}
\NormalTok{t\_90 }\OtherTok{\textless{}{-}} \FunctionTok{qt}\NormalTok{(}\DecValTok{1} \SpecialCharTok{{-}} \FloatTok{0.1}\SpecialCharTok{/}\DecValTok{2}\NormalTok{, df)   }\CommentTok{\# α = 0.1, two{-}tailed}
\NormalTok{t\_95 }\OtherTok{\textless{}{-}} \FunctionTok{qt}\NormalTok{(}\DecValTok{1} \SpecialCharTok{{-}} \FloatTok{0.05}\SpecialCharTok{/}\DecValTok{2}\NormalTok{, df)  }\CommentTok{\# α = 0.05, two{-}tailed}

\CommentTok{\# Standard error of b1}
\NormalTok{SE\_b1 }\OtherTok{\textless{}{-}} \FunctionTok{sqrt}\NormalTok{(MSE }\SpecialCharTok{/} \FunctionTok{sum}\NormalTok{((X }\SpecialCharTok{{-}}\NormalTok{ X\_bar)}\SpecialCharTok{\^{}}\DecValTok{2}\NormalTok{))}

\CommentTok{\# Margin of error}
\NormalTok{ME\_90 }\OtherTok{\textless{}{-}}\NormalTok{ t\_90 }\SpecialCharTok{*}\NormalTok{ SE\_b1}
\NormalTok{ME\_95 }\OtherTok{\textless{}{-}}\NormalTok{ t\_95 }\SpecialCharTok{*}\NormalTok{ SE\_b1}

\CommentTok{\# Output answers}
\FunctionTok{list}\NormalTok{(}
  \AttributeTok{df =}\NormalTok{ df,}
  \AttributeTok{t\_90 =}\NormalTok{ t\_90,}
  \AttributeTok{t\_95 =}\NormalTok{ t\_95,}
  \AttributeTok{SE\_b1 =}\NormalTok{ SE\_b1,}
  \AttributeTok{ME\_90 =}\NormalTok{ ME\_90,}
  \AttributeTok{ME\_95 =}\NormalTok{ ME\_95}
\NormalTok{)}
\end{Highlighting}
\end{Shaded}

\begin{verbatim}
## $df
## [1] 111
## 
## $t_90
## [1] 1.658697
## 
## $t_95
## [1] 1.981567
## 
## $SE_b1
## [1] 0.1144431
## 
## $ME_90
## [1] 0.1898265
## 
## $ME_95
## [1] 0.2267767
\end{verbatim}

\begin{Shaded}
\begin{Highlighting}[]
\CommentTok{\# Question 3: }
\end{Highlighting}
\end{Shaded}

\section{Question 3}\label{question-3}

\begin{enumerate}
\def\labelenumi{\arabic{enumi})}
\setcounter{enumi}{9}
\tightlist
\item
  perform a hypothesis test on the linear impact of X on Y, with a T
  test with a significant value of 0.1. Note: if a question doesn't
  specify the hypothesized value, it is two-sided test against 0. All
  hypothesis problem should include the following component: Ho/Ha
  defined in symbols (β,μ etc.), test statistic (notation and formulas),
  reject region defined on a critical value (p-value computed on a
  probability formula), and conclusion.
\end{enumerate}

\section{Define Hypotheses}\label{define-hypotheses}

H0:β1=0 HA:β1≠0

\section{R Code to Compute t-Statistic and
p-value}\label{r-code-to-compute-t-statistic-and-p-value}

\begin{Shaded}
\begin{Highlighting}[]
\CommentTok{\# Given values}
\NormalTok{b1 }\OtherTok{\textless{}{-}} \FloatTok{0.305}
\NormalTok{SE\_b1 }\OtherTok{\textless{}{-}} \FloatTok{0.0871}
\NormalTok{n }\OtherTok{\textless{}{-}} \DecValTok{113}

\CommentTok{\# Degrees of freedom}
\NormalTok{df }\OtherTok{\textless{}{-}}\NormalTok{ n }\SpecialCharTok{{-}} \DecValTok{2}

\CommentTok{\# Test statistic}
\NormalTok{t\_stat }\OtherTok{\textless{}{-}}\NormalTok{ b1 }\SpecialCharTok{/}\NormalTok{ SE\_b1}

\CommentTok{\# Two{-}tailed p{-}value}
\NormalTok{p\_value }\OtherTok{\textless{}{-}} \DecValTok{2} \SpecialCharTok{*} \FunctionTok{pt}\NormalTok{(}\SpecialCharTok{{-}}\FunctionTok{abs}\NormalTok{(t\_stat), df)}

\CommentTok{\# Critical t{-}value at alpha = 0.1 (two{-}tailed test)}
\NormalTok{t\_crit }\OtherTok{\textless{}{-}} \FunctionTok{qt}\NormalTok{(}\DecValTok{1} \SpecialCharTok{{-}} \FloatTok{0.1}\SpecialCharTok{/}\DecValTok{2}\NormalTok{, df)}

\CommentTok{\# Conclusion}
\NormalTok{reject\_H0 }\OtherTok{\textless{}{-}} \FunctionTok{abs}\NormalTok{(t\_stat) }\SpecialCharTok{\textgreater{}}\NormalTok{ t\_crit}

\CommentTok{\# Output}
\FunctionTok{cat}\NormalTok{(}\StringTok{"Test Statistic (t):"}\NormalTok{, t\_stat, }\StringTok{"}\SpecialCharTok{\textbackslash{}n}\StringTok{"}\NormalTok{)}
\end{Highlighting}
\end{Shaded}

\begin{verbatim}
## Test Statistic (t): 3.501722
\end{verbatim}

\begin{Shaded}
\begin{Highlighting}[]
\FunctionTok{cat}\NormalTok{(}\StringTok{"Degrees of Freedom:"}\NormalTok{, df, }\StringTok{"}\SpecialCharTok{\textbackslash{}n}\StringTok{"}\NormalTok{)}
\end{Highlighting}
\end{Shaded}

\begin{verbatim}
## Degrees of Freedom: 111
\end{verbatim}

\begin{Shaded}
\begin{Highlighting}[]
\FunctionTok{cat}\NormalTok{(}\StringTok{"p{-}value:"}\NormalTok{, p\_value, }\StringTok{"}\SpecialCharTok{\textbackslash{}n}\StringTok{"}\NormalTok{)}
\end{Highlighting}
\end{Shaded}

\begin{verbatim}
## p-value: 0.0006669952
\end{verbatim}

\begin{Shaded}
\begin{Highlighting}[]
\FunctionTok{cat}\NormalTok{(}\StringTok{"Critical t{-}value (alpha = 0.1):"}\NormalTok{, t\_crit, }\StringTok{"}\SpecialCharTok{\textbackslash{}n}\StringTok{"}\NormalTok{)}
\end{Highlighting}
\end{Shaded}

\begin{verbatim}
## Critical t-value (alpha = 0.1): 1.658697
\end{verbatim}

\begin{Shaded}
\begin{Highlighting}[]
\ControlFlowTok{if}\NormalTok{ (reject\_H0) \{}
  \FunctionTok{cat}\NormalTok{(}\StringTok{"Conclusion: Reject the null hypothesis. There is a significant linear relationship.}\SpecialCharTok{\textbackslash{}n}\StringTok{"}\NormalTok{)}
\NormalTok{\} }\ControlFlowTok{else}\NormalTok{ \{}
  \FunctionTok{cat}\NormalTok{(}\StringTok{"Conclusion: Fail to reject the null hypothesis. No significant relationship found.}\SpecialCharTok{\textbackslash{}n}\StringTok{"}\NormalTok{)}
\NormalTok{\}}
\end{Highlighting}
\end{Shaded}

\begin{verbatim}
## Conclusion: Reject the null hypothesis. There is a significant linear relationship.
\end{verbatim}

\section{Defion of terms for the question
3}\label{defion-of-terms-for-the-question-3}

Output Explanation

\begin{verbatim}
t_stat — your test statistic.

p_value — tells how likely your observed value is under the null.

t_crit — cutoff point for rejecting H₀ at α = 0.1.

reject_H0 — TRUE if the result is statistically significant.
\end{verbatim}

\section{Question 4: Using R to compute and interpret questions using
the simple linear regression (SLR)
model:}\label{question-4-using-r-to-compute-and-interpret-questions-using-the-simple-linear-regression-slr-model}

\section{Definition of terms}\label{definition-of-terms}

\begin{verbatim}
Y = length (length of stay)

X = infection (infection risk)
\end{verbatim}

\begin{Shaded}
\begin{Highlighting}[]
\CommentTok{\# Question 4 \# Load data}
\NormalTok{senic }\OtherTok{\textless{}{-}} \FunctionTok{read.csv}\NormalTok{(}\StringTok{"/Users/kwabenabayity/Desktop/MY Purdue/Summer 2025/STATS/Scripts/senic.csv"}\NormalTok{)}

\CommentTok{\# Build simple linear regression model}
\NormalTok{model }\OtherTok{\textless{}{-}} \FunctionTok{lm}\NormalTok{(length }\SpecialCharTok{\textasciitilde{}}\NormalTok{ infection, }\AttributeTok{data =}\NormalTok{ senic)}

\CommentTok{\# View summary of the model}
\FunctionTok{summary}\NormalTok{(model)}
\end{Highlighting}
\end{Shaded}

\begin{verbatim}
## 
## Call:
## lm(formula = length ~ infection, data = senic)
## 
## Residuals:
##     Min      1Q  Median      3Q     Max 
## -3.0587 -0.7776 -0.1487  0.7159  8.2805 
## 
## Coefficients:
##             Estimate Std. Error t value Pr(>|t|)    
## (Intercept)   6.3368     0.5213  12.156  < 2e-16 ***
## infection     0.7604     0.1144   6.645 1.18e-09 ***
## ---
## Signif. codes:  0 '***' 0.001 '**' 0.01 '*' 0.05 '.' 0.1 ' ' 1
## 
## Residual standard error: 1.624 on 111 degrees of freedom
## Multiple R-squared:  0.2846, Adjusted R-squared:  0.2781 
## F-statistic: 44.15 on 1 and 111 DF,  p-value: 1.177e-09
\end{verbatim}

\begin{Shaded}
\begin{Highlighting}[]
\CommentTok{\# a) Standard Error of the Point Estimate (s\{b₁\}) (the SE of the slope coefficient (infection → length))}

\CommentTok{\# Extracting SE of slope (infection)}
\FunctionTok{summary}\NormalTok{(model)}\SpecialCharTok{$}\NormalTok{coefficients[}\StringTok{"infection"}\NormalTok{, }\StringTok{"Std. Error"}\NormalTok{]}
\end{Highlighting}
\end{Shaded}

\begin{verbatim}
## [1] 0.1144431
\end{verbatim}

\begin{Shaded}
\begin{Highlighting}[]
\CommentTok{\# b) Residual Standard Error (RSE) (THis shows the average distance between the observed and predicted Y values.)}

\CommentTok{\# Residual Standard Error}
\FunctionTok{summary}\NormalTok{(model)}\SpecialCharTok{$}\NormalTok{sigma}
\end{Highlighting}
\end{Shaded}

\begin{verbatim}
## [1] 1.624044
\end{verbatim}

\begin{Shaded}
\begin{Highlighting}[]
\CommentTok{\# c) Degrees of Freedom of the Residual (df\_resid)}

\CommentTok{\# Residual degrees of freedom}
\FunctionTok{summary}\NormalTok{(model)}\SpecialCharTok{$}\NormalTok{df[}\DecValTok{2}\NormalTok{]}
\end{Highlighting}
\end{Shaded}

\begin{verbatim}
## [1] 111
\end{verbatim}

\begin{Shaded}
\begin{Highlighting}[]
\CommentTok{\# d) Mean Square Error (MSE) (MSE = SSE / df = RSE²)}

\CommentTok{\# Compute MSE}
\NormalTok{RSE }\OtherTok{\textless{}{-}} \FunctionTok{summary}\NormalTok{(model)}\SpecialCharTok{$}\NormalTok{sigma}
\NormalTok{df\_resid }\OtherTok{\textless{}{-}} \FunctionTok{summary}\NormalTok{(model)}\SpecialCharTok{$}\NormalTok{df[}\DecValTok{2}\NormalTok{]}
\NormalTok{MSE }\OtherTok{\textless{}{-}}\NormalTok{ RSE}\SpecialCharTok{\^{}}\DecValTok{2}
\NormalTok{MSE}
\end{Highlighting}
\end{Shaded}

\begin{verbatim}
## [1] 2.637518
\end{verbatim}

\begin{Shaded}
\begin{Highlighting}[]
\CommentTok{\# e) Standard Deviation of Y (s\_y)}

\CommentTok{\# Standard deviation of Y (length of stay)}
\NormalTok{s\_y }\OtherTok{\textless{}{-}} \FunctionTok{sd}\NormalTok{(senic}\SpecialCharTok{$}\NormalTok{length)}
\NormalTok{s\_y}
\end{Highlighting}
\end{Shaded}

\begin{verbatim}
## [1] 1.911456
\end{verbatim}

\begin{Shaded}
\begin{Highlighting}[]
\CommentTok{\# computing SST}

\CommentTok{\# Total Sum of Squares (SST)}
\NormalTok{n }\OtherTok{\textless{}{-}} \FunctionTok{nrow}\NormalTok{(senic)}
\NormalTok{SST }\OtherTok{\textless{}{-}}\NormalTok{ (n }\SpecialCharTok{{-}} \DecValTok{1}\NormalTok{) }\SpecialCharTok{*}\NormalTok{ s\_y}\SpecialCharTok{\^{}}\DecValTok{2}
\NormalTok{SST}
\end{Highlighting}
\end{Shaded}

\begin{verbatim}
## [1] 409.2104
\end{verbatim}

\end{document}
